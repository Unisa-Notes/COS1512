\providecommand{\main}{../..}
\documentclass[\main/notes.tex]{subfiles}

\begin{document}
	\setcounter{chapter}{7}
	\chapter{Recursion}
		\begin{definition}{Recursion}
			A function that calls itself.
		\end{definition}
		\begin{sidenote}{Recursion Structure}
			For a recursive function to work properly, the following is needed:
			\begin{indentparagraph}
				\begin{description}
					\item[Recursive Case] One or more cases that include one or more recursive calls to the function being defined. These should be `smaller' versions of the task, getting closer to the \concept{base case}.
					\item[Base Case] Also called \concept{stopping case}. One or more cases that include no recursive calls.
				\end{description}
			\end{indentparagraph}
		\end{sidenote}
		In order to keep track of recursion, the computer uses the concept of a \concept{stack}.
		\begin{definition}{Stack}
			A \concept{last-in first-out} memory structure. To start using a stack, an \concept{activation frame} is used.
			\begin{indentparagraph}
				\begin{description}
					\item[Stack Overflow] When the memory available for the stack runs out, likely due to an excessive recursive chain.
				\end{description}
			\end{indentparagraph}
		\end{definition}
		\section{Recursion vs Iteration}
			Recursion is slower, and uses more storage. However, it can produce code that is easier to read.
	\rulechapterend
\end{document}