\providecommand{\main}{../..}
\documentclass[\main/notes.tex]{subfiles}

\begin{document}
	\setcounter{chapter}{5}
	\chapter{Friends and Overloaded Operators}
		\section{Friend Functions}
			\begin{definition}{Friend Function}
				Not a member function of a class, but has access to private members of a class. Must be named as a friend in the class definition.
			\end{definition}
			\begin{codebox}{Declaring friend functions}
				\begin{minted}{cpp}
class ClassName
{
  public:
    friend functionName(functionParams);
};
				\end{minted}
			\end{codebox}
			\begin{sidenote}{Define both accessor functions and friend functions}
				Making every function a friend function is impractical. Therefore both should be used.
			\end{sidenote}
			\begin{sidenote}{Use \texttt{const} and call-by-reference parameters}
				It is more efficicent for a friend function to use a call-by-reference parameter. If the friend function does not change the value of the reference parameters, they can be declared \mintinline{cpp}{const}
				\begin{minted}[bgcolor=noteCode]{cpp}
// An example:
class Money
{
  public:
    friend Money add(const Money& amount1, const Money& amount2);
};

Money add(const Money& amount1, const Money& amount2){}
				\end{minted}
			\end{sidenote}

		\section{Overloading Operators}
			\begin{sidenote}{Operator Overloading}
				A binary operator is a function that is called using a different syntax for listing its arguments. Therefore, these can be overloaded.

				To overload an operator, use the same syntax as overloading a function, but include the keyword \mintinline{cpp}{operator}.
				\begin{minted}[bgcolor=noteCode]{cpp}
Money operator +(const Money& amount1, const Money& amount2){}
				\end{minted}
			\end{sidenote}
			\begin{definition}{Rules for operator overloading}
				\begin{itemize}[nosep]
					\item At least one argument of the resulting overloaded operator must be a class
					\item An overloaded operator can be, but does not have to be, a friend of the class
					\item You cannot create a new operator
					\item You cannot change the number of arguments that an operator takes
					\item You cannot change the precedence of an operator
					\item The following operators \emph{cannot} be overloaded: \mintinline{cpp}{.}, \mintinline{cpp}{::}, \mintinline{cpp}{.*}, \mintinline{cpp}{?:}
					\item Overloading the assignment operator must be done in a different way
				\end{itemize}
			\end{definition}
			\begin{sidenote}{Overloading \texttt{<<} and \texttt{>>}}
				These operators each return a \emph{stream}. Therefore, to overload, the function should return a stream reference.
				\begin{minted}[bgcolor=noteCode]{cpp}
friend ostream& operator << (ostream& outs, const Class& name);
friend istream& operator >> (istream& ins, const Class& name);
				\end{minted}
			\end{sidenote}
			\begin{sidenote}{The `Big Three'}
				\begin{itemize}[nosep]
					\item Copy constructor
					\item Destructor
					\item \texttt{=} Operator
				\end{itemize}
			\end{sidenote}
			\begin{codebox}{Overloading the assignment operator}
				\begin{minted}{cpp}
class ClassName
{
  public:
    void operator =(const ClassName& rightSide);
}
				\end{minted}
			\end{codebox}

		\section{The \texttt{this} pointer}
			\begin{definition}{\texttt{this} pointer}
				Used to refer to the calling object. Not the name of the calling object, but rather a ponter that points to the calling object.
				\begin{minted}[bgcolor=definitionCode]{cpp}
this->variable;
				\end{minted}
			\end{definition}

		\section{Overloading operators as member functions}
			\begin{codebox}{Example}
				\begin{minted}{cpp}
class Money
{
  public:
    Money operator +(const Money& amount2);
}
// As opposed to
class Money
{
  public:
    friend Money operator +(const Money& amount1,
                            const Money& amount2);
}
				\end{minted}
			\end{codebox}
	\rulechapterend
\end{document}