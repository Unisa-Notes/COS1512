\providecommand{\main}{../..}
\documentclass[\main/notes.tex]{subfiles}

\begin{document}
	\setcounter{chapter}{2}
	\chapter{Strings and Vectors}
		\section{C-Strings}
			\begin{definition}{C-string}
				An array of characters terminated with \mintinline{cpp}{'\0'}, also known as the \concept{null character}.
			\end{definition}
			A string literal is stored using a c-string.

			When declaring a c-string as an array, remember to add $1$ to the length, in order to include the termination character.
			\begin{codebox}{Declaring a C-string}
				\begin{minted}{cpp}
char ArrayName[Max_C_string_size + 1];
				\end{minted}
			\end{codebox}
			\begin{sidenote}{Initialising a c-string}
				A c-string can be initialised when declared, using a string literal.
				\begin{minted}[bgcolor=noteCode]{cpp}
char myMessage[20] = "Hi there.";
				\end{minted}
			\end{sidenote}
			\begin{sidenote}{An array of characters is not a c-string!}
				The following two lines of code are \emph{not} equivalent.
				\begin{minted}[bgcolor=noteCode]{cpp}
char cString[] = "abc";
char charArray[] = {'a', 'b', 'c'};
				\end{minted}
			\end{sidenote}
			\noindent As a c-string is an array, you can index it the same way you would a normal array. Be careful to not destroy the \mintinline{cpp}{'\0'} using that.

			\subsection{C-string Assignment}
				\begin{sidenote}{C-strings cannot use \mintinline{cpp}{=} after initialised}
					As a c-string is an array, after the c-string is initialised, you cannot use \mintinline{cpp}{=} to assign a new value.
					\begin{minted}[bgcolor=noteCode]{cpp}
char aString[10];
aString = "Hello"; // illegal!
					\end{minted}
					Instead, to assign a value to a c-string variable, use:
					\begin{minted}[bgcolor=noteCode]{cpp}
char aString[10];
strcpy(aString, "Hello");
					\end{minted}
					This, however, does not check whether the new string exceeds the size. To ensure a copied-over string does not exceed the size, use \mintinline{cpp}{strncpy}
					\begin{minted}[bgcolor=noteCode]{cpp}
char bString[10];
strcpy(bString, aString, 9);
					\end{minted}
					At max, $9$ characters will be copied.
				\end{sidenote}

			\subsection{Comparing C-strings}
			\begin{sidenote}{C-strings cannot use \mintinline{cpp}{==} to compare}
				To compare c-strings, use \mintinline{cpp}{strcmp}. This function returns true if the strings \emph{do not} match.
				\begin{minted}[bgcolor=noteCode]{cpp}
if (strcmp(cString1, cString2))
  cout << "The strings are NOT the same.";
else
  cout << "The strings are the same.";
				\end{minted}
				C-strings are compared using \concept{lexicographic order}. It compares character by character. If the characters are not the same, it stops. If the character in the first string is greater than the character in the second string, then a positive number is returned. If the character in the first string is less, a negative number is returned. If it gets to the end of the string, a $0$ is returned.
			\end{sidenote}
			The functions used for c-string comparison and assignment are defined in the \mintinline{cpp}{<cstring>} library.
			\pagebreak

			\subsection{C-String Library}
				\begin{definition}{\texttt{<cstring>} library}
					\begin{center}
						\begin{tblr}{>{\raggedright}X>{\raggedright}X>{\raggedright}X}
							Function & Description & Cautions\\
							\midrule
							\mintinline[breaklines]{cpp}{void strcopy(cstring target, cstring src);} & Copies the value of \mintinline{cpp}{src} into \mintinline{cpp}{target}. & Doesn't check whether \mintinline{cpp}{target} is large enough.\\
							\mintinline[breaklines]{cpp}{void strcopy(cstring target, cstring src, int limit);} & Copies the value of \mintinline{cpp}{src} into \mintinline{cpp}{target}, for at most \mintinline{cpp}{limit} characters. & Not implemented in all C++ versions.\\
							\mintinline[breaklines]{cpp}{void strcat(cstring target, cstring src);} & Concatenates \mintinline{cpp}{src} to the end of \mintinline{cpp}{target}. & Doesn't check whether \mintinline{cpp}{target} is large enough.\\
							\mintinline[breaklines]{cpp}{void strncat(cstring target, cstring src, int limit);} & Concatenates \mintinline{cpp}{src} to the end of \mintinline{cpp}{target} for at most \mintinline{cpp}{limit} characters. & Not implemented in all C++ versions.\\
							\mintinline[breaklines]{cpp}{int strlen(cstring src);} & Returns the number of characters in \mintinline{cpp}{src}. & The number does not include the termination character \mintinline{cpp}{'\0'}\\
							\mintinline[breaklines]{cpp}{int strcmp(cstring str1, cstring str2);} & Returns $0$ if \mintinline{cpp}{str1} has the same characters as \mintinline{cpp}{str2}. Returns a negative value if the character in \mintinline{cpp}{str1} is less than the character in \mintinline{cpp}{str2}. Returns a positive value if the character in \mintinline{cpp}{str1} is greater than the character in \mintinline{cpp}{str2}. Compares lexigraphically. & If the cstrings are the same, $0$, which translates to \emph{false}, is returned.\\
							\mintinline[breaklines]{cpp}{int strncmp(cstring str1, cstring str2, int limit);} & Same as \mintinline{cpp}{strcmp}, except only \mintinline{cpp}{limit} number of characters are compared. & Not implemented in all C++ versions.
						\end{tblr}
					\end{center}
				\end{definition}

			\subsection{C-String Conversions}
				\begin{definition}{C-String to Number}
					Use \mintinline{cpp}{atoi}, \mintinline{cpp}{atof}, and \mintinline{cpp}{atol} to convert a c-string to an integer, float, or long respectively.
					\begin{minted}[bgcolor=definitionCode]{cpp}
#include <cstdlib>
using namespace std;

int x = atoi("123");
float y = atof("45.67");
long z = atol("89101112");
					\end{minted}
				\end{definition}

		\section{Strings}
			\begin{definition}{\texttt{string}}
				A class that allows programmers to treat strings as a basic data type. To use, include the \mintinline{cpp}{<string>} library.

				Allows one to use string values and variables similarly to that of a basic type 
				\begin{itemize}
					\item \mintinline{cpp}{=} can be used for assignment
					\item \mintinline{cpp}{+} can be used for concatenation
				\end{itemize}
			\end{definition}
			\begin{codebox}{String Constructors}
				\begin{minted}{cpp}
#include <string>
using namespace std;

string phrase;
string noun("words");
// equivalent to
string noun = "words";
				\end{minted}
			\end{codebox}
			\begin{sidenote}{String to C-String}
				To convert a string to a cstring, use the function \mintinline{cpp}{.c_str()}.
				\begin{minted}[bgcolor=noteCode]{cpp}
string name = "Hello world!";
char nameCString[] = name.c_str();
				\end{minted}
			\end{sidenote}

		\section{Vectors}
			\begin{definition}{Vector}
				An array that can grow and shrink in size, while the program is running. A vector is a \concept{template class}.
			\end{definition}
			\begin{codebox}{Declaring Vectors}
				\begin{minted}{cpp}
#include <vector>
using namespace std;

vector<Base_Type> name;

// Examples:
vector<int> v1;
vector<char> v2;
				\end{minted}
			\end{codebox}
			\begin{sidenote}{Square brackets not used to initialise}
				Like arrays, square brackets can be used to access and set the values at a particular index.
				\begin{minted}[bgcolor=noteCode]{cpp}
v[i] = 42;
				\end{minted}

				\noindent However, if that index has not been initialised, the index \emph{cannot} be initialised this way. Instead, use \mintinline{cpp}{vector.push_back()}.
				\begin{minted}[bgcolor=noteCode]{cpp}
#include <vector>
using namespace std;

vector<int> v;
v.push_back(1);
v.push_back(2);
v.push_back(3); // v is now [1, 2, 3]
				\end{minted}

				\noindent A vector can be initialised in the same way as an array:
				\begin{minted}[bgcolor=noteCode]{cpp}
vector<char> v = {'a', 'b', 'c'};
				\end{minted}
			\end{sidenote}
			\begin{sidenote}{Size of a vector}
				To determine the size of a vector, use \mintinline{cpp}{vector.size()}. This returns an \emph{unsigned int}, instead of an int.
				\begin{minted}[bgcolor=noteCode]{cpp}
vector<int> sample;
for (unsigned int i = 0; i < sample.size(); i++)
	cout << sample[i] << endl;
				\end{minted}
			\end{sidenote}
	\rulechapterend
\end{document}