\providecommand{\main}{../..}
\documentclass[\main/notes.tex]{subfiles}

\begin{document}
	\setcounter{chapter}{9}
	\chapter{Templates}
		\section{Function Templates}
			\begin{definition}{Function Template}
				Define a function that can work on different types.
				\begin{minted}[bgcolor=definitionCode]{cpp}
template<class T>
				\end{minted}
				This is called the \concept{template prefix}, and needs to appear before each template function.
				\vspace{10px}

				\noindent An example:
				\begin{minted}[bgcolor=definitionCode]{cpp}
template<class T>
void swapValues(T& variable1, T& variable2)
{
    T temp;

    temp = variable1;
    variable1 = variable2;
    variable2 = temp;
}
				\end{minted}
			\end{definition}
			\begin{sidenote}{More than one type parameter}
				To declare more than one type, do the following:
				\begin{minted}[bgcolor=noteCode]{cpp}
template<class T1, class T2>
				\end{minted}
			\end{sidenote}
			\begin{sidenote}{Template Functions in Header and Implementation Files}
If there are template functions in the header \emph{and} implementation files, then the \texttt{.cpp} file needs to also be included in the main file.
				\begin{minted}[bgcolor=noteCode]{cpp}
#include "class.h"
#include "class.cpp"
				\end{minted}
			\end{sidenote}

		\section{Data Templates}
			Use the same syntax as for function templates. Declare this template \emph{before} the class, and then before each implemented function.
			\begin{codebox}{Data template}
				\begin{minted}[bgcolor=]{cpp}
template<class> T
class Pair
{
    public:
        Pair();
        Pair(T val1, T val2);
        void setElement(int position, T value);
        T getElement(int position) const;
    private:
        T first;
        T second;
}

template<class T>
Pair<T>::Pair(T val1, T val2)
       : first(val1), second(val2) {}

template<class T>
T Pair<T>::getElement(int position) const {...}
				\end{minted}


			\end{codebox}

	\rulechapterend
\end{document}