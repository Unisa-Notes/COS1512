\providecommand{\main}{../..}
\documentclass[\main/notes.tex]{subfiles}

\begin{document}
	\setcounter{chapter}{1}
	\chapter{I/O Streams}
		\section{Streams}
			\begin{definition}{Stream}
				A C++ object that is a flow of characters (or other form of data). If data is coming in, it is called an \concept{input stream} (\texttt{ifstream}). If data is going out of the program, it is called an \concept{output stream} (\texttt{ofstream}).
			\end{definition}
			\begin{codebox}{Defining streams}
				\begin{minted}{cpp}
#include <fstream>
using namespace std;

ifstream inStream;
ofstream outStream;
			\end{minted}
		\end{codebox}
		\begin{codebox}{Connecting files to a stream}
			\begin{minted}{cpp}
inStream.open("infile.dat");
// or
ifstream inStream("infile.dat");
				\end{minted}
			\end{codebox}
			\begin{sidenote}{A file has two names}
				Everu file has an \concept{external file name}, the actual name of the file on the system. Every file also has a stream that you associate with it, and then you refer to the file using the stream.
			\end{sidenote}
			\pagebreak
			\begin{codebox}{Stream Example}
				\inputminted[]{cpp}{\subfix{code/01_example.cpp}}
			\end{codebox}

		\section{Classes and Objects}
			\begin{definition}{Object}
				A variable that has functions as well as data associated with it.
			\end{definition}
			Different objects can have different member functions with the same names, or different names.
			\begin{definition}{Class}
				A type whose variables are objects.
			\end{definition}

		\pagebreak
		\section{File Names as Input}
			A \texttt{c-string} needs to be sent to the stream for the input file.
			\begin{codebox}{File Name Input Example}
				\begin{minted}{cpp}
// Using c-string
char file_name[16];
cin >> file_name;
ifstream inStream(file_name);

// Using string
string file_name;
cin >> file_name;
ifstream inStream(file_name.c_str());

				\end{minted}
			\end{codebox}

		\section{Processing Files Tips}
			\begin{sidenote}{Check whether the file opened successfully}
				\begin{itemize}
					\item Use \mintinline{cpp}{stream.fail()} -- a boolean function to check whether a file has successfully opened.
					\item If a file was not opened successfully, you can \mintinline{cpp}{exit} the program. \mintinline{cpp}{exit} is part of the \mintinline{cpp}{<cstdlib>} library.
				\end{itemize}
				\begin{minted}[bgcolor=noteCode]{cpp}
#include <fstream>
#include <cstdlib>
using namespace std;

int main()
{
    ifstream inStream;
    inStream.open("input.dat");
    if (inStream.fail())
    {
        cout << "Unable to open file.\n";
        exit(1);
    }
    return 0;
}
				\end{minted}
			\end{sidenote}
		\pagebreak

		\section{Data Files vs Text Files}
			\begin{itemize}
				\item A \concept{data file} should be read as \texttt{string}s, \texttt{int}s, etc. This contains data that should be processed as a unit.
				\item A \concept{text file} should be processed character by character.
			\end{itemize}
			\begin{codebox}{Reading Data Files}
				\inputminted[]{cpp}{\subfix{code/02_data_files.cpp}}
			\end{codebox}
			\pagebreak
			\begin{codebox}{Reading Text Files}
				\inputminted[]{cpp}{\subfix{code/03_text_files.cpp}}
			\end{codebox}
			\begin{itemize}[nosep]
				\item The function \mintinline{cpp}{.get} reads one character from an instream
				\item The function \mintinline{cpp}{.put} places one character in an outstream
				\item The function \mintinline{cpp}{.eof} determines if the next character is the end of file marker.
			\end{itemize}

		\section{Streams as arguments to functions}
			A stream parameter must \emph{always} be call-by-reference.
			\mint[bgcolor=codeColor]{cpp}{void function(ifstream& stream);}

	\rulechapterend
\end{document}