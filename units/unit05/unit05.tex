\providecommand{\main}{../..}
\documentclass[\main/notes.tex]{subfiles}

\begin{document}
	\setcounter{chapter}{4}
	\chapter{Defining Classes}
		\section{Classes}
			\begin{codebox}{Basic Class}
				\begin{minted}{cpp}
class ClassName
{
  public:
    Member_1;
    Member_N;
  protected:
    Member_N+1;
    Member_N+2;
  private:
    Member_N+3;
    Member_N+4;
}
				\end{minted}
			\end{codebox}
			\begin{definition}{Accessors and Mutators}
				Also called \concept{getters and setters}. Used to access and set private member variables.
				\begin{minted}[bgcolor=definitionCode]{cpp}
class ClassName
{
  public:
    void setName(string newName); // mutator
    string getName(); // accessor
  private:
    string name;
};
				\end{minted}
				\pagebreak
				\begin{minted}[bgcolor=definitionCode]{cpp}
void ClassName::setName(string newName)
{
  name = newName;
}
string ClassName::getName()
{
  return name;
}
				\end{minted}
			\end{definition}
			\subsection{Constructors}
				\begin{definition}{Constructor}
					Used to initialise some or all member variables of an object when it is declared. A member function that is automatically called when an object of that class is declared.
					\begin{itemize}[nosep]
						\item Must have the same name as the class.
						\item Cannot return a value.
					\end{itemize}
					\begin{minted}[bgcolor=definitionCode]{cpp}
class ClassName
{
	public:
		ClassName(string name); // constructor
	private:
		string name;
};
ClassName::ClassName(string name)
{
	this->name = name;
}
					\end{minted}
				\end{definition}
				\begin{sidenote}{Constructor Initialization Section}
					The constructor can have a section before the function that is used to set the member variables of the class.
					\begin{minted}[bgcolor=noteCode]{cpp}
class ClassName
{
	public:
		ClassName(int set1, int set2);
	private:
		int member1;
		int member2;
};
ClassName::ClassName(int set1, int set2)
				: member1(set1), member2(set2) {}
					\end{minted}
				\end{sidenote}
				\begin{codebox}{Calling a Constructor}
					\begin{minted}{cpp}
Object = ConstructorName(ConstructorArguments);
account1 = BankAccount(200, 3.5);

ClassName ObjectName(ConstructorArguments);
BankAccount account1(100, 2.3);
					\end{minted}
				\end{codebox}
				\begin{sidenote}{Always include a default constructor}
					If an object is created without arguments, the default constructor is called. So this should be included in the class definition, if another constructor is created.
					\begin{minted}[bgcolor=noteCode]{cpp}
class ClassName
{
	public:
		ClassName(string name);
		ClassName(); // default constructor
	private:
		string name;
};
					\end{minted}
				\end{sidenote}
				\begin{definition}{Copy Constructor}
					A constructor that has one parameter that is of the same type of the class. This parameter must be a call-by-reference paramneter.
				\end{definition}
			\subsection{Destructors}
				\begin{definition}{Destructor}
					A member function that is called automatically when an object of the class passes out of scope. Used to delete any dynamic variables that an object uses.
					\begin{itemize}[nosep]
						\item Must have the same name as the class, preceded by a \mintinline{cpp}{~}.
						\item Cannot return a value.
					\end{itemize}
					\begin{minted}[bgcolor=definitionCode]{cpp}
class ClassName
{
  public:
    ~ClassName(); // destructor
  private:
    int *value; // hold a dynamic array
};
ClassName::~ClassName()
{
  delete [] value; // delete dynamic array
}
					\end{minted}
				\end{definition}

		\section{Abstract Data Types}
			\begin{definition}{Data Type}
				A collection of values together with a set of basic operations defined on those types.
			\end{definition}
			\begin{definition}{Abstract Data Type (ADT)}
				Programmers who use the type do not have access to the details of how the values and operations are implemented.
			\end{definition}
			\begin{sidenote}{How to Write an ADT}
				\begin{enumerate}[nosep]
					\item Make all the member variables private members of the class.
					\item Make each of the basic operations needed a public member function.
					\item Make any helping functions private member functions.
				\end{enumerate}
			\end{sidenote}
			\begin{sidenote}{Interface}
				Explains how to use an ADT in the program -- header file.
			\end{sidenote}
			\begin{sidenote}{Implementation}
				Explains how the interface is realised as C++ code.
			\end{sidenote}

		\section{Inheritance}
			Define a more general class, and then derive a more specific class from that general class.
			\begin{definition}{Derived Classes}
				A class that inherits functionality from a more general class.

				\noindent To delare a derived class:
				\begin{minted}[bgcolor=definitionCode]{cpp}
class DerivedName : public ParentName
{
  public:
    DerivedName(int member1, int member2);
    // other constructors
}
				\end{minted}
				The derived class gets every public member of the general (or parent) class. Only new functionality needs to be defined.
			\end{definition}

	\rulechapterend
\end{document}